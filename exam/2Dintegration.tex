\documentclass{article}
\usepackage{fullpage}
\usepackage{mathtools}
\usepackage{amssymb}
\usepackage{graphicx}
\usepackage{listings}
\title{A short note on 2D integrator implementation}
\author{K. R. ~Pedersen, 201709218}
\date{}
\begin{document}
\lstset{language=C}
\maketitle

\begin{abstract}
In this note I shortly describe the ideas behind my two-dimensional integrator implementation.
\end{abstract}

\section{Introduction}
The task is to numerically calculate two-dimensional integrals of the form
\begin{equation}
	I = \int_a^b {\mathrm d x} \int_{d(x)}^{u(x)} {\mathrm d y} \: f(x,y),
	\label{integral}
\end{equation}
by consecutively applying an adaptive one-dimensional integrator along each of the two dimensions. Let
\begin{equation}
	F(x):= \int_{d(x)}^{u(x)} {\mathrm d y} \: f(x,y),
	\label{define F}
\end{equation}
so that
\begin{equation}
	I=\int_a^b {\mathrm d x} \: F(x).
	\label{I short}
\end{equation}
The integral I is estimated by a weighted sum of the form
\begin{equation}
	I \approx \sum_i w_iF(x_i),
	\label{I app}
\end{equation}
where the function values $F(x_i)$ are given by the integral
\begin{equation}
	F(x_i) = \int_{d(x_i)}^{u(x_i)} {\mathrm d y} \: f(x_i, y),
	\label{Fxi def}
\end{equation}
and estimated as a weighted sum in a similar fashion
\begin{equation}
	F(x_i) \approx \sum_j \omega_j f(x_i, y_j).
	\label{Fxi app}
\end{equation}
Both Eq. (\ref{I app}) and (\ref{Fxi app}) are estimated using the one-dimensional recursive adaptive integrator from the coursework. 



\end{document}
